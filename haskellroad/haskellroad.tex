\documentclass[11pt]{article}
\usepackage{amssymb}


    \title{\textbf{The Haskell Road to Logic, Maths, and Programming - Exercises \& Notes}}
    \author{}
    \date{}
    
    \addtolength{\topmargin}{-3cm}
    \addtolength{\textheight}{3cm}
\begin{document}

\maketitle
\thispagestyle{empty}

\section*{Chapter 2}
\subsection*{2.1 Logical Connectives and their Meanings}
\subsubsection*{Connectives}

\begin{tabular}{|l|l|l|} \hline
     {} & \textbf{symbol} & \textbf{name}\\ \hline
	and & $\land$ & conjunction\\ \hline
	or & $\lor$ & disjunction\\ \hline
	not & $\neg$ & negation\\ \hline
	if--then & $\Rightarrow$ & implication\\ \hline
	if, and only if & $\iff$ & equivalence\\ \hline
\end{tabular}
\subsubsection*{Negation Truth Table}
\begin{tabular}{|l|l|l|}
\hline
	\textbf{P} & $\neg$\textbf{P}\\
\hline
	t & f\\
\hline
    f & t\\
\hline
\end{tabular}
\subsubsection*{Conjunction Truth Table}
\begin{tabular}{|l|l|l|}
\hline
	\textbf{P} & \textbf{Q} & \textbf{P} $\land$ \textbf{Q}\\
\hline
	t & t & t\\
\hline
	t & f & f\\
\hline
	f & t & f\\
\hline
	f & f & f\\
\hline
\end{tabular}

\subsubsection*{Inclusive Or Truth Table}
\begin{tabular}{|l|l|l|}
\hline
	\textbf{P} & \textbf{Q} & \textbf{P} $\lor$ \textbf{Q}\\
\hline
	t & t & t\\
\hline
	t & f & t\\
\hline
	f & t & t\\
\hline
	f & f & f\\
\hline
\end{tabular}

\subsubsection*{Exclusive Or Truth Table}
\begin{tabular}{|l|l|l|}
\hline
	\textbf{P} & \textbf{Q} & \textbf{P} $\veebar$ \textbf{Q}\\
\hline
	t & t & f\\
\hline
	t & f & t\\
\hline
	f & t & t\\
\hline
	f & f & f\\
\hline
\end{tabular}

\subsubsection*{Implication Truth Table}
\begin{tabular}{|l|l|l|}
\hline
	\textbf{P} & \textbf{Q} & \textbf{P} $\Rightarrow$ \textbf{Q}\\
\hline
	t & t & t\\
\hline
	t & f & f\\
\hline
	f & t & t\\
\hline
	f & f & t\\
\hline
\end{tabular}
\\\\
P is the \textit{antecendent} of the implication and Q is the \textit{consequent}. \\ Example: For every natural number $n$, \begin{center}$5 < n \Rightarrow 3 < n$.\end{center} The implication is \emph{true} if
\begin{itemize}
\item both antecedent and consequent are false ($n = 2$),
\item antecedent false, consequent true ($n=4$), and
\item both antecedent and consequent are true ($n=6$).
\end{itemize}
The implication is false when the antecedent is true and the consequent is false. 

Implication in Haskell:
\begin{verbatim}
infix 1 ==>

(==>) :: Bool -> Bool -> Bool
x ==> y = (not x) || y
\end{verbatim}

The implication is false when $x \land \lnot y$, so the opposite of this is $\lnot x \lor y$. The \texttt{1} sets the binding power or precedence of the operation.

A decleration of an infix operator together with an indication of its binding power is called a \textit{fixity operation}.

It is also possible to give a direct definition:
\begin{verbatim}
(==>) :: Bool -> Bool -> Bool
True  ==> x = x
False ==> x = True
\end{verbatim}

\end{document}

