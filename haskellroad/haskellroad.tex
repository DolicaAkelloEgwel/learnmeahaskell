\documentclass[11pt]{article}
\usepackage{amssymb}


    \title{\textbf{The Haskell Road to Logic, Maths, and Programming - Exercises \& Notes}}
    \author{}
    \date{}
    
    \addtolength{\topmargin}{-3cm}
    \addtolength{\textheight}{3cm}
\begin{document}

\maketitle
\thispagestyle{empty}

\section*{Chapter 2}
\subsection*{2.1 Logical Connectives and their Meanings}
\subsubsection*{Connectives}

\begin{tabular}{|l|l|l|} \hline
     {} & \textbf{symbol} & \textbf{name}\\ \hline
	and & $\land$ & conjunction\\ \hline
	or & $\lor$ & disjunction\\ \hline
	not & $\neg$ & negation\\ \hline
	if--then & $\Rightarrow$ & implication\\ \hline
	if, and only if & $\iff$ & equivalence\\ \hline
\end{tabular}
\subsubsection*{Negation Truth Table}
\begin{tabular}{|l|l|l|}
\hline
	\textbf{P} & $\neg$\textbf{P}\\
\hline
	t & f\\
\hline
    f & t\\
\hline
\end{tabular}
\subsubsection*{Conjunction Truth Table}
\begin{tabular}{|l|l|l|}
\hline
	\textbf{P} & \textbf{Q} & \textbf{P} $\land$ \textbf{Q}\\
\hline
	t & t & t\\
\hline
	t & f & f\\
\hline
	f & t & f\\
\hline
	f & f & f\\
\hline
\end{tabular}

\subsubsection*{Inclusive Or Truth Table}
\begin{tabular}{|l|l|l|}
\hline
	\textbf{P} & \textbf{Q} & \textbf{P} $\lor$ \textbf{Q}\\
\hline
	t & t & t\\
\hline
	t & f & t\\
\hline
	f & t & t\\
\hline
	f & f & f\\
\hline
\end{tabular}

\subsubsection*{Exclusive Or Truth Table}
\begin{tabular}{|l|l|l|}
\hline
	\textbf{P} & \textbf{Q} & \textbf{P} $\veebar$ \textbf{Q}\\
\hline
	t & t & f\\
\hline
	t & f & t\\
\hline
	f & t & t\\
\hline
	f & f & f\\
\hline
\end{tabular}

\subsubsection*{Implication Truth Table}
\begin{tabular}{|l|l|l|}
\hline
	\textbf{P} & \textbf{Q} & \textbf{P} $\Rightarrow$ \textbf{Q}\\
\hline
	t & t & t\\
\hline
	t & f & f\\
\hline
	f & t & t\\
\hline
	f & f & t\\
\hline
\end{tabular}
\\\\
P is the \textit{antecendent} of the implication and Q is the \textit{consequent}. \\ Example: For every natural number $n$, \begin{center}$5 < n \Rightarrow 3 < n$.\end{center} The implication is \emph{true} if
\begin{itemize}
\item both antecedent and consequent are false ($n = 2$),
\item antecedent false, consequent true ($n=4$), and
\item both antecedent and consequent are true ($n=6$).
\end{itemize}
The implication is false when the antecedent is true and the consequent is false. 

Implication in Haskell:
\begin{verbatim}
infix 1 ==>

(==>) :: Bool -> Bool -> Bool
x ==> y = (not x) || y
\end{verbatim}

The implication is false when $x \land \lnot y$, so the opposite of this is $\lnot x \lor y$. The \texttt{1} sets the binding power or precedence of the operation.

A decleration of an infix operator together with an indication of its binding power is called a \textit{fixity operation}.

It is also possible to give a \textbf{direct} definition:
\begin{verbatim}
(==>) :: Bool -> Bool -> Bool
True  ==> x = x
False ==> x = True
\end{verbatim}

\paragraph{Trivially True Implications.} Implications with the antecedent false and the consequent true are true regardless of the parameters. These are dubbed \textit{trivially true} implications.

My name is Napolean $\Rightarrow$ the decimal expansion of $\pi$ contains the sequence 7777777. \newline
The decimal expansion of $\pi$ contains 7777777 $\Rightarrow$ Strawberries are red

\paragraph{Converse and Contrapostion.} 

The \emph{converse} of an implication $P \Rightarrow Q$ is $Q \Rightarrow P$. The \emph{contrapostion} of the implication $P \Rightarrow Q$ is $\lnot Q \Rightarrow \lnot P$. The converse of a true implication does not need to be true, but its contrapostion is true iff the implication is true.

\paragraph{Necessary and Sufficient Conditions.} To say that $P$ is a \textit{sufficient condition} for $Q$ is to say that \textbf{the presence of $P$ guarantees the presence of $Q$}. In other words, it is impossible to have $P$ without $Q$. The statement $P$ is called a \emph{sufficient} condition for $Q$ because $P$ being true implies that $Q$ is true, but $P$ being false does not always imply that $Q$ is false. \newline $Q$ is \emph{necessary} for $P$ because it is impossible to have $P$ without $Q$. Example: \textbf{If it is raining, then there are clouds in the sky.}

An implication $P \Rightarrow Q$ can be expressed in a number of ways:
\begin{enumerate}
\item if $P$, then $Q$,
\item $Q$ if $P$,
\item $P$ only if $Q$,
\item $Q$ whenever $P$,
\item $P$ is sufficient for $Q$,
\item $Q$ is necessary for $P$.
\end{enumerate}

\paragraph{Equivalence}
The expression $P \iff Q$ is called the \textit{equivalence} of $P$ and $Q$. $P$ and $Q$ are members of the equivalence. An equivalence $P \iff Q$ amounts to the conjuction of the implications $P \Rightarrow Q$ and $Q \Rightarrow P$. If $P \iff Q$ is true then $P$ is a \textit{necessary} and \textit{sufficient} condition for $Q$.

\begin{tabular}{|l|l|l|}
\hline
	\textbf{P} & \textbf{Q} & \textbf{P} $\iff$ \textbf{Q}\\
\hline
	t & t & t\\
\hline
	t & f & f\\
\hline
	f & t & f\\
\hline
	f & f & t\\
\hline
\end{tabular}

The truth table for exclusive or is the same as the one for $\lnot (P \iff Q)$. So we can use the code below for the Haskell implementation of the function \texttt{<+>} (exclusive or).

\begin{verbatim}
(<+>) :: Bool -> Bool -> Bool
x (<+>) y = x /= y
\end{verbatim}

Using Haskell to find the truth value of a formula: 
$$\lnot \land ((P \Rightarrow Q) \iff \lnot (Q \land \lnot P)).$$

\begin{verbatim}
p = True
q = False
formula1 = (not p) && (p ==> q) <=> not (q && (not p))
\end{verbatim}

Loading the formula:
\begin{verbatim}
TAMO> formula1
False
\end{verbatim}

\subsection*{2.2 Logical Validity and Related Notions}

\textbf{Logical validities} are propositonal formulas that receive the value $t$ no matter what the value of the "arguments." Examples of logical validites are: $P \Rightarrow P, P \lor \lnot P, P \Rightarrow (Q \Rightarrow P)$.
If an expression contains $n$ letters then there are $2^n$ possible distributions of the truth values between these letters. If all calculated values are equal to $t$ then the expression, by definition, is a validity.

A propositional/Boolean/truth function can be used to check the validity of a propositon given different arguments. 

\begin{verbatim}
formula2 p q = ((not p) && (p ==> q) <=> not (q && (not p)))
\end{verbatim}

This has the type \texttt{Bool -> Bool -> Bool} whereas \texttt{formula1} is simply \texttt{Bool}.

\end{document}

